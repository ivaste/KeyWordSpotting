\documentclass[conference]{IEEEtran}
\ifCLASSINFOpdf
\else
\fi
\hyphenation{op-tical net-works semi-conduc-tor}
\begin{document}

\title{Keyword Spotting}
\author{\IEEEauthorblockN{Stefano Ivancich}
\IEEEauthorblockA{Department of\\Information Engineering\\
University of Padova, Italy\\
Email: stefano.ivancich@studenti.unipd.it}
\and
\IEEEauthorblockN{Luca Masiero}
\IEEEauthorblockA{Department of\\Information Engineering\\
University of Padova, Italy\\
Email: luca.masiero.8@studenti.unipd.it}}

\maketitle

\begin{abstract}
Lorem ipsum dolor sit amet, consectetur adipiscing elit. Nulla est purus, ultrices in porttitor
in, accumsan non quam. Nam consectetur porttitor rhoncus. Curabitur eu est et leo feugiat
auctor vel quis lorem. Ut et ligula dolor, sit amet consequat lorem. Aliquam porta eros sed
velit imperdiet egestas. Maecenas tempus eros ut diam ullamcorper id dictum libero
tempor. Donec quis augue quis magna condimentum lobortis. Quisque imperdiet ipsum vel
magna viverra rutrum. Cras viverra molestie urna, vitae vestibulum turpis varius id.
Vestibulum mollis, arcu iaculis bibendum varius, velit sapien blandit metus, ac posuere lorem
nulla ac dolor. Maecenas urna elit, tincidunt in dapibus nec, vehicula eu dui. Duis lacinia
fringilla massa. Cum sociis natoque penatibus et magnis dis parturient montes, nascetur
ridiculus mus. Ut consequat ultricies est, non rhoncus mauris congue porta. Vivamus viverra
suscipit felis eget condimentum. Cum sociis natoque penatibus et magnis dis parturient
montes, nascetur ridiculus mus. Integer bibendum sagittis ligula, non faucibus nulla volutpat
vitae. Cum sociis natoque penatibus et magnis dis parturient montes, nascetur ridiculus mus.  
In aliquet quam et velit bibendum accumsan. Cum sociis natoque penatibus et magnis dis
parturient montes, nascetur ridiculus mus. Vestibulum vitae ipsum nec arcu semper
adipiscing at ac lacus. Praesent id pellentesque orci.
\end{abstract}

\IEEEpeerreviewmaketitle

\section{Introduction}
The goal of keyword spotting is to detect a relatively small set of predefined keywords in a stream of user utterances, usually in the context of an intelligent agent on a mobile phone or a consumer "smart home" device. With the rapid development of mobile devices, speech-related technologies are becoming increasingly popular. For example, Google offers the ability to search by voice on Android phones, while personal assistants such as Google Now, Apple’s Siri, Microsoft’s Cortana and Amazon’s Alexa, all uti-
lize speech recognition to interact with these systems. Google has enabled a fully hands-free speech recognition experience, known as “Ok Google”, which continuously listens for specific keywords to initiate voice input. This keyword spotting (KWS) system runs on mobile devices, and therefore must have
a small memory footprint and low computational power.

Keyword spotting can be used to detect “command triggers” such as “hey Siri”, which provide explicit cues for interactions directed at the device. It is additionally desirable that such models have a small footprint (for example, measured in the number of model parameters) so they can be deployed on low power and performance-limited devices. In recent years, neural networks have been shown to provide effective solutions to the small-footprint keyword spotting problem. Research typically focuses on a tradeoff be-
tween achieving high detection accuracy and having a small footprint.

In this work we focus on recurrent neural networks (RNNs), a class of models that has been successfully applied to small-footprint keyword spotting in recent years.

SPIEGARE PERCHè LE RNN VANNO BENE PER QUESTO TASK

DA INSERIRE DA QUALCHE PARTE: Recurrent neural networks represent a ground-breaking advance in deep learning that has allowed researchers to successfully train deeper networks.

For this project we use the recently-released Google Speech Command Dataset as our benchmark.

AGGIUNGERE QUALCOSA DA INTRO DI Attention-Based Models ... SOLO SE SERVE

DA METTERE ALLA FINE DELL'INTRODUZIONE: This paper is structured as follows. In Section 2 we give an overview of the... Section 3 presents... The experimental setup is described in Section 4, while results comparing... are presented in Section 5. Finally, Section 6 concludes the paper and discusses future works. 
\hfill June 29, 2020

\subsection{Subsection Heading Here}
Subsection text here.


\subsubsection{Subsubsection Heading Here}
Subsubsection text here.

\section{Related work}
The first system similar to a modern ASR was built in the 1952 by researchers at the Bell laboratories and was able to recognize numerical digits from speech using formats of the input audio. These are a concentration of the acoustic energy around a particular frequency in the input file wave. For the next thirty years, various researchers developed devices capable of recognizing vowels and consonants using different types of features like \textit{phonemes}, until the introduction, in the mid 1980s of the Hidden Markov Models (HMM). 

This approach represented a significant shift from the simple pattern recognition methods that were based on templates (and a spectral distance measure), to a statistical method for speech processing and was possible thank to the incredible advances in the computer computational power obtained(?) in those years. In recent years, however, the HMMs faced the challenge of the introduction of Deep Learning and several architechtures that work well with these type of problems like Convolutional Neural Networks (CNN) shift-invariant in the data representation domain, and Recurrent Neural Networks (RNN) and their ability to store information.
\section{Model implementation}
This section describes our base model and its variants. All code necessary to replicate our experiment has been made oper source in our Github repository\footnote{https://github.com/ivaste/KeyWordSpotting}.
\section{Evaluation}
DA MODIFICARE We evaluated our models using Google’s Speech Commands Dataset, which was released in August 2017 under a Creative Commons license\footnote{https://research.googleblog.com/2017/08/launching-speech-commands-dataset.html}. The dataset contains 65,000
one-second long utterances of 30 short words by thousands of
different people, as well as background noise samples such as
pink noise, white noise, and human-made sounds.
\section{Conclusion}
PRENDI SPUNTO DA QUESTO: This paper describes the application of deep residual learning and dilated convolutions to the keyword spotting problem. Our work is enabled by the recent release of Google’s Speech Commands Dataset, which provides a common benchmark for this task. Previously, related work was mostly incomparable because papers relied on private datasets. Our work establishes new, state-of-the-art, open-source reference models on this dataset that we encourage others to build on. For future work, we plan to compare our CNN-based approaches with an emerging family of models based on recurrent architectures. We have not undertaken such a study because there do not appear to be publicly-available reference implementations of such models, and the lack of a common benchmark makes comparisons difficult. The latter problem has been addressed, and it would be interesting to see how
recurrent neural networks stack up against our approach.

\section*{Acknowledgment}
All the experiments were conducted using Keras and blabla libraries.

The authors would like to acknowledge the support of...


\begin{thebibliography}{1}
\bibitem{1}
Douglas Coimbra de Andrade, Sabato Leo, Martin Loesener Da Silva Viana, Christoph Bernkopf. \textit{A neural attention model for speech command recognition}. arXiv:1808.08929

\bibitem{2}
Jan Chorowski, Dzmitry Bahdanau, Dmitriy Serdyuk, Kyunghyun Cho, Yoshua Bengio. \textit{Attention-Based Models for Speech Recognition}. arXiv:1506.07503

\bibitem{3}
Sainath, Tara N. / Parada, Carolina (2015). \textit{Convolutional neural networks for small-footprint keyword spotting}, In INTERSPEECH-2015, 1478-1482.

\bibitem{4}
Raphael Tang, Jimmy Lin. \textit{Deep Residual Learning for Small-Footprint Keyword Spotting}. arXiv:1710.10361

\bibitem{5}
Alon, G.. \textit{Key-Word Spotting-The Base Technology for Speech Analytics}.

\bibitem{6}
Pete Warden. \textit{Speech Commands: A Dataset for Limited-Vocabulary Speech Recognition}.
\end{thebibliography}
\end{document}


